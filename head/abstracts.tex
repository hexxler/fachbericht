%\begingroup
%\let\cleardoublepage\clearpage


% English abstract
\cleardoublepage
\chapter*{Abstract}
\markboth{Abstract}{Abstract}
\addcontentsline{toc}{chapter}{Abstract} % adds an entry to the table of contents
% put your text here
\par Die Entwicklung von Spielen hat seit den ersten Versuchen in den 1950er und 60er bis heute einen grossen Wandel mitgemacht. Über die Jahre hinweg wurden viele Engines veröffentlicht, die als Versprechen das vereinfachen der
ganzen Entwicklungsprozedur mit sich brachten. Diese Produkte abstrahieren Entwicklungen, die bei Spielen immer wieder anzutreffen sind: die Grafische Darstellung von Objekten, Laufzeitmanagement und vieles mehr. Da die Auswahl an solchen Engines ist gross, und dieser Bericht beschränkt sich auf drei grosse: \emph{Unity, Unreal Engine} und \emph{CryEngine}. Diese überschneiden sich in vielen Aspekten: Sie alle bieten einen Kostenlosen einstieg, eigene Anleitungen und wurden bereits für die Entwicklung mehrerer Titel genutzt, die allesamt mehrere Millionen USD Gewinn einbrachten.

\par Wie man als Entwickler bei diesen drei Engines nun Logik, Shaders und dergleichen Programmiert ist nicht ganz identisch. So offeriert Unity als einzige Engine eine Visuelle Optionen, Shaders zu programmieren, während für Logiken selber alle drei Engines einen Visuellen Editor anbieten. Dies zeigt sich aber auch bei den Programmiersprachen. In Unity wird \emph{C\#} genutzt, während \emph{Unreal Engine} und \emph{CryEngine} auf C++ beharren. Dies ist in erster Linie nichts negatives - C++ erlaubt es dem Erfahrenen Entwickler, tiefgründiger mit dem Speicher umzugehen und so bessere Optimierungen zu realisieren. Gleichzeitig bedeutet dies aber für Neueinsteiger, das man leichter Fehler einbaut, die im Zweifelsfall zu spät erkannt werden.

\par Alle 3 Optionen bieten eigene \emph{Asset Stores} an. Diese werden genutzt, um eine Vielzahl an Fertigprodukte zu vertreiben. Von Modellen, Animationen, Texturen bis hin zu Sound und Plugins ist alles erhältlich. Auch hier bieten die Engine-Hersteller kostenlose Pakete an, um den Einstieg in die Spielentwicklung zu vereinfachen.

\par Wie bei den Anleitungen hat aber auch hier Unity die Nase vorn. Zunächst bietet Unity die grösste Anzahl Anleitungen/Kurse an von allen drei. Die einzelnen Optionen kommen in diversen Umfängen dar. Von einfachen Anleitungen, wo man rudimentär den \emph{A*-Pathfinder} implementiert, bis hin zum bauen eines Vollwertigen Spieles mit Offizielen Assets. Sollten die Unity-Eigenen Optionen nicht helfen, schaffen Drittanbieter Aushilfe. Hier findet man, im Vergleich zu \emph{Unreal Engine} und \emph{CryEngine} eine schier unersättliche Anzahl Optionen. 

%\endgroup			
%\vfill
