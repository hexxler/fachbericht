% !TeX program = lualatex

\documentclass[a4paper]{article}

% Packages %%%%%%%%%%%%%%%%%%%%%%%%%%%%%%%%%%%%%%%
\usepackage{relsize}
\usepackage{geometry}
\usepackage{titlepic}
%\usepackage[utf8]{inputenc} % inputenc is not required if lualatex is used
\usepackage{authblk}
\usepackage{fancyhdr}
\usepackage[urlcolor = accent_light,linktocpage]{hyperref}
\usepackage{tocloft}
\usepackage{lipsum}
\usepackage{graphicx}
\usepackage{float}
\usepackage{pdfpages}
\usepackage{comment}
\usepackage[normalem]{ulem}
\usepackage[none]{hyphenat}
\usepackage{sectsty}
\usepackage{listings}
\usepackage{tabulary}
\usepackage[backend=bibtex, style=ieee, citestyle=ieee]{biblatex}
\usepackage{subfiles}
\usepackage{fontspec}
\usepackage[nopostdot, acronym, toc]{glossaries}
\usepackage[nottoc, notlot, notlof]{tocbibind}
\usepackage{parskip}
\usepackage{enumitem}
\usepackage{svg}
\usepackage{csquotes}
\usepackage{hyperref}


% Overall settings %%%%%%%%%%%%%%%%%%%%%%%%%%%%%%%
\graphicspath{{images/}}
\svgpath{{images/svg/}}
\pagestyle{fancy}
\fancyhf{}


\input{corporate_design/color_def.tex}
\input{corporate_design/emphasis_commands.tex}


\defaultfontfeatures{Mapping=tex-text,Scale=MatchLowercase}
\setmainfont{Arial}[
	SizeFeatures={Size=11}
]

\setmonofont{Nimbus Mono}


\hypersetup{
	colorlinks=true,
	linkcolor=accent_light,
	citecolor=accent_light,
}

\sectionfont{\color{accent_light}}
\subsectionfont{\color{accent_normal}}
\subsubsectionfont{\color{accent_dark}}


% Appendices %%%%%%%%%%%%%%%%%%%%%%%%%%%%%%%%%%%%%
\makeglossaries
\setacronymstyle{sc-short-long}  
\loadglsentries{appendices/glossaries}

\addbibresource{appendices/references.bib}


%% Document settings %%%%%%%%%%%%%%%%%%%%%%%%%%%%%
% content settings
\newcommand{\project}{Projekt Hexxle}
\newcommand{\outline}{Fachbericht}


\title{
	\Huge{}\color{accent_normal}\includesvg[width=\linewidth]{hexxle_logo_1}\\ 
	\vspace{2cm}
	\large{}\color{accent_dark}\textbf{\outline}
}


\author{
	Deniz Akca
	\and
	Dennis Bannerman
	\and
	Adriana de Sao José Martins
	\and
	Ricardo Monteiro Simoes
	\and
	Andrew Surber
}
\affil{ZHAW - Zurich}


\rhead{\color{content_light}\includesvg[width=2cm]{hexxle_logo_2}}
\lhead{\color{content_light}\project\ $\vert$ \textit{\leftmark}}

\rfoot{\color{content_light}-\thepage-}


\renewcommand{\cftsecleader}{\cftdotfill{\cftdotsep}}
\renewcommand{\baselinestretch}{1.5}


% Define titlepage layout
\makeatletter
\def\@maketitle{%
	\newpage
	\null
	\vskip 1cm%
	\begin{center}%
		\let \footnote \thanks
		{\LARGE \@title \par}%
		\vskip 2cm%
		{\large
			\lineskip .25em%
			\begin{tabular}[t]{c}%
				\@author
			\end{tabular}\par}%
		\vfill%
		{\large \@date}
	\end{center}%
	\par
	\vskip 1.5em
}
\makeatother

% \todo{Draft watermark, should be removed for final}
\usepackage{draftwatermark}
\SetWatermarkText{Draft}
\SetWatermarkColor[gray]{0.9}
\SetWatermarkAngle{60}
\SetWatermarkScale{20}
% 

% Improvements for package todonotes, remove for final
\usepackage{xargs}                      % Use more than one optional parameter in a new commands
\usepackage[colorinlistoftodos,prependcaption,textsize=tiny]{todonotes}
\newcommandx{\unsure}[2][1=]{\todo[linecolor=red,backgroundcolor=red!25,bordercolor=red,#1]{#2}}
\newcommandx{\change}[2][1=]{\todo[linecolor=blue,backgroundcolor=blue!25,bordercolor=blue,#1]{#2}}
\newcommandx{\info}[2][1=]{\todo[linecolor=green,backgroundcolor=green!25,bordercolor=green,#1]{#2}}
\newcommandx{\improvement}[2][1=]{\todo[linecolor=violet,backgroundcolor=violet!25,bordercolor=violet,#1]{#2}}
\newcommandx{\thiswillnotshow}[2][1=]{\todo[disable,#1]{#2}}
%

% Document structure %%%%%%%%%%%%%%%%%%%%%%%%%%%%%
\begin{document}
	\sloppy
	\pagenumbering{roman}
	\color{content_normal}
	
	\begin{titlepage}
		\maketitle
		\thispagestyle{empty}
	\end{titlepage}

	\subfile{sections/abstract}
	\tableofcontents
	\newpage
	
	\pagenumbering{arabic}

	
	%% Einleitung %%	
	\subfile{sections/einleitung}
	
	%% Resultate %%
	\subfile{sections/resultate}
	
	%% Diskussion und Ausblick %%
	\subfile{sections/diskussion}
	
	%% Appendix %%
	\appendix
	% \section{...}
	
	
	\newpage

	%% Glossar %%
	\printglossary[type=\acronymtype]
	\printglossary
	
	%% Literaturverzeichnis %%
	\newpage
	\printbibliography[heading=bibintoc]
	
	\newpage
	\listoftodos[Notes]
\end{document}

% Der  Fachartikelim  Rahmen desPSIT4-Kursesschliesst  das  Gruppenprojekt dieses  Semesters in schriftlicher  Form  ab. Dieser  wissenschaftlich gehaltene  Berichtdient  dazu,  das  fertige  Projekt  in einer einschlägigenFachzeitschrift   zu   präsentieren.   Es   handelt   sich   also   um   eine   Experten-Experten-Kommunikation, in  der  ein  korrekter  und  konsequenter  Umgang  mit  Fachterminologie ebenso wichtig ist wie eine sachliche und neutrale Formulierung.

% FormaleAnforderungenDer hier  geforderte Fachartikelumfasst ca. sieben  Seiten, wozuDeckblatt  und  Literaturverzeichnis nicht  gerechnet werden.Ein Inhaltsverzeichnis ist nicht nötig, aber alle Seiten sind  nummeriertund enthalten   konsequente   Kopfzeilen.   Schrift   Arial,   11Punkte,   Blocksatz   mit   leserichtiger   Silben-trennung, Zeilenabstand 1.5.

% Bewertung
% Inhaltliche Qualität: Wissenschaftliche Beschreibungen, Recherchen zum Stand der Technik,konzi-se Zusammenfassung, Beurteilung der Qualität und Relevanz der Ergebnisse (gemässLeit-faden Berichtstrukturund Grundlagenblatt zumAbstract).
% Formales:  Aufbau, Titelstruktur, Text/Bild-Integration,  korrekte  und  vollständige  Formalien(gemässZitierleitfadenIEEE).
% Sprache: Korrektheit, Kohärenz der Aussagen, wissenschaftlicher Stil, unpersönliches Formulieren.