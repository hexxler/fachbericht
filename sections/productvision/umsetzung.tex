\documentclass[../main.tex]{subfiles}

\begin{document}
	\section{Umsetzung}
	\todo[inline]{Festlegen der Hauptfeatures und der Ziele für Definition of Done}
	Im Folgenden wird der Begriff Benutzer*innen verwendet, welcher synonym zu den im Kapitel \nameref{section:Spielregeln} genannten Spieler*innen ist.
	
	\subsection{Hauptfeatures}
	\label{section:Hauptfeatures}
	\par Im Rahmen der Produktvision beschränken sich die Hauptfeatures auf eines \gls{mvp}. Für die Realisierung eines MVP ist die Umsetzung der folgenden Bedingungen zwingend:
	\begin{itemize}
		\item Hauptmenü
		\begin{itemize}
			\item Partie starten
			\item Partie beenden
			\item Spiel beenden
		\end{itemize}
		\item Hauptspiel
		\begin{itemize}
			\item Spielfeld
			\item Möglichkeit, Plättchen zu legen
			\item Plättchenstapel
			\item Punkteanzeige
			\item Hauptmenü-Button
		\end{itemize}
		\item Spielmechanik
		\begin{itemize}
			\item Punktemechanik (ohne Intervalle/Plättchenmarkt)
			\item 3 Plättchentypen
			\item 1 Plättchenart (kreisförmig bzw. direkt anliegend)
			\item 1 Plättchenverhalten (benachbarte Plättchen bleiben liegen)
		\end{itemize}
	\end{itemize}

	\subsection{Definition of Done}
	\label{section:DefinitionOfDone}
	\par Die im Kapitel \nameref{section:Hauptfeatures} genannten Features werden aufgrund der im folgenden genannten Kriterien als vollständig umgesetzt angesehen. Die Abnahme geschieht durch den Product Owner.
	
	\subsubsection{Hauptmenü}
	\par Nachdem das Spiel gestartet oder während einer Partie der Hauptmenü-Button gedrückt wurde, befinden sich die Benutzer*innen im Hauptmenü. Im Hauptmenü sind drei Funktionen implementiert, wovon zwei von den Benutzer*innen nach Spielstart ausgewählt werden. Die dritte Option `Partie beenden´ ist deaktiviert, solange die Benutzer*innen sich nicht in einer Partie befinden.
	\begin{itemize}
		\item Partie starten: Die Benutzer*innen können mittels dieser Funktion eine neue Partie starten. Nachdem sie sich für diese Option entschieden haben, werden die Benutzer*innen auf die Ansicht des Hauptspiels weitergeleitet. Das Hauptspiel wird den Regeln entsprechend initialisiert.
		\item Spiel beenden: Die Benutzer*innen können das Spiel mittels dieser Funktion beenden. Nachdem sie sich für diese Option entschieden haben, wird das Spiel beendet. Die Benutzer*innen befinden sich anschliessend zurück auf der Standardoberfläche ihres Systems.
		\item Partie beenden: Die Benutzer*innen können mittels dieser Funktion die laufende Partie beenden. Nachdem sie sich für diese Option entschieden haben, wird die Partie beendet und die Benutzer*innen werden zurück ins Hauptmenü geführt.
	\end{itemize}

	\subsubsection{Hauptspiel}
	\par Wird eine neue Partie gestartet, so befinden sich die Benutzer*innen anschliessend in der Ansicht des Hauptspiels. Sichtbar müssen hierbei das Spielfeld, die Punkteanzeige, der Plättchenstapel und der Hauptmenü-Button sein.
	\par Der Plättchenstapel wurde mit zufälligen Plättchen den \nameref{section:Spielregeln} entsprechend initialisiert. Das Spielfeld ist leer und die Punkteanzeige auf Null.
	\par Es ist den Benutzer*innen möglich, durch einen Klick auf den Hauptmenü-Button ins Hauptmenü zu gelangen. Es ist den Benutzer*innen ausserdem möglich, das oberste Plättchen des Plättchenstapels gemäss den \nameref{section:Spielregeln} auf dem Spielfeld zu platzieren.
	\par Sind die Bedingungen für das Ende einer Partie gemäss den \nameref{section:Spielregeln} gegeben, werden die Benutzer*innen zurück ins Hauptmenü geführt.
	
	\subsubsection{Spielmechanik}
	\par Es wurden drei verschiedene Plättchentypen vollständig inklusive ihrer Beziehungen zum eigenen und den übrigen Plättchentypen  implementiert.
	\par Es wurde eine Plättchenart implementiert. Diese Plättchenart bedingt, dass alle an das Plättchen direkt anliegenden Plättchen für die Punkteberechnung einkalkuliert werden.
	\par Es wurde ein Plättchenverhalten implementiert. Dieses Plättchenverhalten bedingt, dass unter der Plättchenart berücksichtigte Plättchen unverändert liegen bleiben.
	\par Für jedes platzierte Plättchen werden den Benutzer*innen die korrekte Anzahl Punkte zugeschrieben.
	
	\subsubsection{Zusatz}
	\par Zusätzlich zu den bereits genannten funktionalen Kriterien, werden folgende Abnahmekriterien definiert:
	\begin{itemize}
		\item Der Code wurde einem \gls{peerreview} unterzogen.
		\item Der Code wurde getestet und hat die Tests bestanden.
		\item Der Code ist dokumentiert.
	\end{itemize}
\end{document}