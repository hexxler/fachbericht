\documentclass[../main.tex]{subfiles}

\begin{document}
    % In dieser Sektion sollen die qualitativen Attribute und nicht funktionalen Anforderungen definiert werden.
    % Folgende Fragen sollten beantwortet werden:
    %   •Welche Qualitätsattribute soll die Software/das System erfüllen?
    %   •Sind die Qualitatsatribute SMART (specific, measurable, achievable, relevant, timely)?
    %   •Gibt es Attribute, die üblicherweise als gegeben betrachtet werden explizit ausgenommensind?
    %   •Sind alle Attribute und Anforderungen realistisch?
    % Hier einige typische Qualitätsattribute/Anforderungen, die definiert werden sollten (nicht immer alle erforderlich):
    %   •Performance (Latenz, Durchsatz),
    %   •Skalierbarkeit (Daten-, Umsatz-Volumen),
    %   •Verfügbarkeit (erforderliche Uptime, erlaubte Downtime, maintenance windows,...),
    %   •Security (authentication, authorization, confidentiality, ...),
    %   •Erweiterbarkeit,
    %   •Flexibilität,
    %   •Nachverfolgbarkeit (Auditing),
    %   •Monitoring & Management,
    %   •Zuverlässigkeit,
    %   •Failover/Disaster Recovery Ziele,
    %   •Interoperabilität,
    %   •rechtliche und regulatorische Anforderungen,
    %   •Internationalisierung,
    %   •Zugänglichkeit,
    %   •Benutzerfreundlichkeit, ...
    %
    % Eine einfache Auflistung mit präziser Definition, die keinen Spielraum für Interpretation lässt, soll-te ausreichen.
    % Es sind nicht immer alle Anforderungen sinnvoll bzw. erforderlich.
    % Dieses Kapitel muss in jedem Software-Guidebook enthalten sein
	\section{Nicht-funktionale Anforderungen}
	\todo[inline]{todo}
\end{document}