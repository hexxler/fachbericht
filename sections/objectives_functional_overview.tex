\documentclass[../main.tex]{subfiles}

\begin{document}
    % Auch wenn es nicht das Ziel des Guidebooksist, die Funktionalität im Detail zu beschreiben, ist es sinnvoll die Hauptziele und Hauptfunktionen kurz zu benennen.
    % Die Details werden dann in den User-Stories beschrieben.
    % Referenzen auf bestehende Dokumentation sollen verwendet werden (Link zu User Stories)
    % Dieses Kapitel sollte folgende Fragen beantworten:
    %   •Ist allen Teilnehmern klar, was das System, die Software macht?
    %   •Welche Funktionen sind essentiell für den Erfolg des Produkts
    %   •Wann ist es ein Erfolg? Welche Ziele müssen erreicht werden? (quantitative technische, wirtschaftliche oder organisatorische Ziele)
    % Dieses Kapitel muss in jedem Software-Guidebook enthalten sein
    \section{Funktionale Anforderungen}
    Ziel des Projektes ist die Umsetzung einer Spielidee mittels \gls{unity}. Zum aktuellen Zeitpunkt hat sich das Team auf die in den folgenden Unterkapiteln aufgeführten Punkte festgelegt, die anhand der in \gls{jira} aufgeführten User-Stories umgesetzt werden. Nähere Details zur Spielidee finden sich 
    \subsection{Stories}
    \begin{itemize}
    	\item Der User muss das Spiel starten können.
    	\item Der User muss eine Partie starten können.
    	\item Der User muss eine Partie gemäss den Spielregeln zu Ende spielen können.
    	\begin{itemize}
    		\item Der User muss Plättchen aus seiner Hand auf das Spielfeld legen können.
    		\item Der User muss für gelegte Plättchen Punkte erhalten.
    		\item Wenn der User Plättchen legt, wird der leere Platz auf der Hand durch ein Plättchen vom Stapel aufgefüllt, sofern der Stapel noch Plättchen beinhält.
    		\item Der User muss durch Punkte neue Plättchen erhalten können.
    		\item Wenn der User keine Plättchen mehr hat, wird die Partie beendet.
    	\end{itemize}
    	\item Der User kann die laufende Partie zu pausieren.
    	\item Der User kann ein Optionsmenü aufrufen.
    	\item Der User kann ein Infomenü aufrufen.
    	\item Der User muss das Spiel beenden können.
    \end{itemize}
	\par \todo[inline]{Link zu passenden JIRA Stories}

	\subsection{Akzeptanzkriterien}

	\subsection{Erfolgskriterien}
\end{document}