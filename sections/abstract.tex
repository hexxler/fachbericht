\documentclass[../main.tex]{subfiles}

\begin{document}
	\section*{Abstract}
	\todo[inline]{Write abstract.}
	
	\newpage
\end{document}
% Abstracts, welche sich auf Fachtexte beziehen, die sich an den „harten Wissenschaften“ ori-entieren,  erfordern  meist  einen  bestimmten  formalen  Aufbau.  Die  nachfolgende  Beschrei-bung ist modellhaft als Schreibhilfe gedacht –etwa für technische Berichte oder Bachelorar-beiten. 
% 1.Einleitung:  Definiert  die  Problematik  und  begründet  die  Relevanz  der  Untersuchung. Die  Situation  (eine  problematische  Situation  oder  technische  Problematik),  die  wis-senschaftliche/fachliche  Untersuchungs-Fragestellung  (folgt  logischerweise  aus  der festgestellten Problematik) wird kurz beschrieben
% 2.Methodische Einordnung der Arbeit:  Art,  Datenbasis  und  Ziel  der  Arbeit.  Die  Unter-suchungsmethode (Umfrage, Analyse, Versuch, Test etc.), das untersuchte „Material“ und das dabei verfolgte Zielwird thematisiert.
% 3.Vorgehen:  Informiert  exemplarisch  über  die  Untersuchungsanlage.Typische  Frage-stellungen werden genannt oder eine Versuchsanordnung oder ein Testbeispiel wird exemplarisch beschrieben.
% 4.Ergebnis: Beschreibt die wichtigsten Resultate, Erkenntnisse, offene Fragen. Die Er-gebnisse werden aufgeführt und dabei wird auf die unter 1. aufgeführte Fragestellung Bezug genommen (konnte sie beantwortet werden, gibt es offene Punkte?).