\chapter{Diskussion}
\markboth{Diskussion}{Diskussion}
\addcontentsline{toc}{chapter}{Diskussion}

In diesem Kapitel wird auf die im Kapitel \ref{chap:resultate} referenzierten Daten Bezug genommen, sowie darauf basierend eine Entscheidung gefällt.

\section{Kostenvergleich}

Bei den Kosten haben Neueinsteiger Glück. Sie können ohne jegliche Gebühren die Engines selber ausprobieren, und bis zu einem gewissen Punkt Gebührenfrei entwickelte Spiele verkaufen. Da dieser Bericht darauf zielt, eine Entscheidungsgrundlage für Anfänger zu bieten, werden Kosten bei grösseren Jahresgewinne ignoriert. Dies basiert auf der Überlegung, dass das Grundlegende Wissen für die Entwicklung und Gestaltung eines Spiels - analog zum Programmieren - frei transferierbar ist. Wenn man also eine gute Basis hat, kann man anhand der Spezialanforderungen für das zu Entwickelnde Projekt die richtige Engine auswählen.

\section{Programmiersprachen}

Die Programmiersprache \emph{C++} is im Vergleich zwei mal Eingetragen, bei \emph{Unreal Engine} und \emph{CryEngine}. Dies hat einen einfachen Grund. C++ erlaubt es dem Programmierer, bei Bedarf den Speicher selbst zu Verwalten, anstatt das dies dem \emph{Garbage Collector} zu überlasse. Mit den richtigen Konzept kann dies spürbare Unterschiede bei der Leistung mit sich bringen. Entsprechend steigt auch der Komplexitätsgrad der Sprache. Gemäss StackOverflow\cite{stackoverflow_stack_2020} und GitHub\cite{github_state_2020} ist \emph{C\#} bei der Nutzung \emph{C++} mehrere Plätze voraus. Einen Grund dafür könnte sein, das eine besondere Ähnlichkeit zu \emph{Java}\footnote{Java schneidet bei beiden Umfragen besser ab als C\#} existiert\footnote{Wird nicht umsonst auch 'Microsoft Java' genannt}. Die meisten Lehrinstitute bringen als Einstiegssprache \emph{Java} bei - die hohe Verbosität erleichtert es, sich zurecht zu finden.

\par Zusätzlich zu den gewöhnlichen Programmiersprachen, bieten alle drei noch Visuelle Optionen an. Diese erlauben es, Logiken und Abläufe mit simplen Verbindungen zu erstellen. Dies hilft zusätzlich, wenn man einsteigt. 

\section{Anleitungen/Kurse}
