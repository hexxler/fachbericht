\chapter*{Resultate}
\markboth{Resultate}{Resultate}
\addcontentsline{toc}{chapter}{Resultate}

In diesem Kapitel werden die Verfügbaren Produkte diverser Kriterien bewertet.

\section{Daten}
\par Es gibt eine Vielzahl von nutzbaren Produkten, die im Jahr \the\year relevant sind. Um eine Nachvollziehbare Analyse zu erstellt benötigt es an brauchbaren Zahlen und Werten. Bezüglich Nutzerzahlen, Marktanteil sowie Aktivitätsbereich sind nur Angaben für \emph{Unity} erhältlich.\footnote{Künstler umfasst jegliche Personen die mit Unity selbst arbeiten}

\begin{table}[h]
	\centering
	\caption[Unity Zahlen]{Unity Zahlen\cite{unity_unity_2021}}
	\label{tab:unityzahlen}
	\begin{tabular}{c|c}
		\toprule
		Monatliche aktive Künstler	& 1.5 Mio\\
		Marktanteil & 50\%+ \\
		Aktive Länder & 190+ \\
		\bottomrule 
	\end{tabular}
\end{table}


\par Um trotzdem einen Vergleich sowie eine Auswahl zwischen den möglichen Produkten zu finden, wird ein Vergleich der Kostenstruktur erstellt.

\begin{table}[h]
	\centering
	\caption[Kostenstruktur]{Kostenstruktur}
	\label{tab:kostenstruktur}
	\begin{tabular}{{p{0.3\linewidth} | p{0.15\linewidth} | p{0.25\linewidth} | p{0.15\linewidth}}}
		\toprule
		Bedingung & Unity\cite{technologies_unity_2021} & Unreal Engine\cite{unreal_engine_unreal_2021} & CryEngine\cite{cryengine_cryengine_2021} \\
		\midrule
		0\$+ Jahresgewinn & 0\% & 0\% & 0\% \\
		5'000\$+ Jahresgewinn & 0\% & 0\% & 5\% \\
		100'000\$+ Jahresgewinn & 369\$ /Jahr /Nutzer & 0\% & 5\% \\
		200'000\$+ Jahresgewinn & 1'656\$ /Jahr /Nutzer & 0\% & 5\% \\
		200'000\$+ Jahresgewinn und mehr als 10 Nutzer & 183,33\$ /Monat /Nutzer & 0\% & 5\% \\
		1'000'000\$+ Totaler Gewinn /Veröffentlichten Titel & Siehe oben gem. Anzahl Nutzer & 5\% auf den Gewinn nach 1'000'000\$ & 5\% \\
		\bottomrule 
	\end{tabular}
\end{table}

\par Ein wichtiger Faktor zu jeder Engine ist die Verwendete Programmiersprache bzw. die möglichen Programmiertools. Diese reichen von simplen Sprachen bis hin zu Visuellen Editoren. Je nach Programmiersprache ist so die mögliche Komplexität und Fehlerquote (Beispielsweise Memory-Management mit C++) ein Hindernis für Einsteiger. Gleichzeitig bieten Visuelle Darstellungen fürs Programmieren Einstiegshilfen. So können generische Lösungen erstellt werden, ohne grosse Programmierkenntnisse zu haben.

\begin{table}[h]
	\centering
	\caption[Programmiersprachen]{Programmiersprachen}
	\label{tab:programmiersprachen}
	\begin{tabular}{{p{0.3\linewidth} | p{0.15\linewidth} | p{0.25\linewidth} | p{0.15\linewidth}}}
		\toprule
		Type & Unity\cite{unity_introduction_2021}\cite{noauthor_unity_nodate} & Unreal Engine\cite{unreal_engine_blueprint_2021} & CryEngine\cite{cryengine_cryengine_2021-1} \\
		\midrule
		Programmiersprache & C\# & C++ & C++/Lua/C\# \\
		Andere Programmierarten & ShaderGraph, Visual Scripting  & Blueprints & FlowGraph \\
		\bottomrule 
	\end{tabular}
\end{table}

\par Die Nutzung eines Produkts benötigt auch immer eine gewisse Ausbildung im Umgang damit. Heutzutage gibt es verschiedene Optionen. Einerseits gibt es Offizielle Anleitungen/Schulungen, andererseits bieten oft Dritte Anleitungen an, wie man die Tools verwendet.\footnote{Bei der Datenerhebung von Eigenen Anleitungen/Kurse wurden nicht einzelne Videos gezählt, sondern Ansammlungen davon. So zählt z.B. bei der CryEngine die Anleitung "Getting Started" als 1,  während "Animation" mit den 3 eigenen Subkategorien als 3 Einheiten zählt.}\footnote{Datenerfassen für "Dritte" wurde hier mittels Google Suche erstellt. Zur Verwertung wurden nur Sucheinträge unter der Kategorie "Videos" erfasst. Die Suche selbst wurde mit "x tutorial" durchgeführt, wobei x $\in$ \{"Unity", "Unreal Engine", "CryEngine"\}. Zusätzlich wurden 10\% der Maximalen Anzahl Suchergebnisse Subtrahiert, um allfällige Falscheinträge zu vermindern.}

\begin{table}[h]
	\centering
	\caption[Anleitungen / Kurse]{Anleitungen / Kurse}
	\label{tab:anleitungen/kurse}
	\begin{tabular}{{p{0.3\linewidth} | p{0.15\linewidth} | p{0.25\linewidth} | p{0.15\linewidth}}}
		\toprule
		Herausgeber & Unity\cite{unity_learn_2021} & Unreal Engine\cite{unreal_engine_unreal_2021-1} & CryEngine\cite{cryengine_cryengine_2021-2} \\
		\midrule
		Eigene & 1'566 & 103 & 38 \\
		Drittanbietern & 17'370'000  & 663'300 & 30'960 \\
		\bottomrule 
	\end{tabular}
\end{table}
