\chapter{Resultate}
\markboth{Resultate}{Resultate}
\addcontentsline{toc}{chapter}{Resultate}
\label{chap:resultate}

\section{Daten}
\par Es gibt eine Vielzahl von nutzbaren Produkten, die im Jahr \the\year relevant sind. Um eine Nachvollziehbare Analyse zu erstellt benötigt es an brauchbaren Zahlen und Werten. Bezüglich Nutzerzahlen, Marktanteil sowie Aktivitätsbereich sind nur Angaben für \emph{Unity} erhältlich.\footnote{Künstler umfasst jegliche Personen die mit Unity selbst arbeiten}

\begin{table}[h]
	\centering
	\caption[Unity Zahlen]{Unity Zahlen\cite{unity_unity_2021}}
	\label{tab:unityzahlen}
	\begin{tabular}{c|c}
		\toprule
		Monatliche aktive Künstler	& 1.5 Mio\\
		Marktanteil & 50\%+ \\
		Aktive Länder & 190+ \\
		\bottomrule 
	\end{tabular}
\end{table}


\par Um trotzdem einen Vergleich sowie eine Auswahl zwischen den möglichen Produkten zu finden, werden mehr Daten benötigt. Hierfür wird das Feld als Sicht eines Neueinsteigers betrachtet. Die hier als wichtig gewichteten Punkte sind:

\begin{itemize}
	\item Kosten
	\item Optionen für das Programmieren
	\item Wissensaneignung
	\item Handhabung von Problemen
\end{itemize}

\newpage

Die Kosten sind gut auffindbar und bieten eine gute, initiale Übersicht.

\begin{table}[h]
	\caption[Kosten]{Kosten}
	\label{tab:kosten}
	\begin{tabular}{{p{0.3\linewidth} | p{0.15\linewidth} | p{0.20\linewidth} | p{0.13\linewidth}}}
		\toprule
		Bedingung & Unity\cite{technologies_unity_2021} & Unreal Engine\cite{unreal_engine_unreal_2021} & CryEngine\cite{cryengine_cryengine_2021} \\
		\midrule
		0\$+ Jahresgewinn & 0\% & 0\% & 0\% \\
		5'000\$+ Jahresgewinn & 0\% & 0\% & 5\% \\
		100'000\$+ Jahresgewinn & 369\$ /Jahr /Nutzer & 0\% & 5\% \\
		200'000\$+ Jahresgewinn & 1'656\$ /Jahr /Nutzer & 0\% & 5\% \\
		200'000\$+ Jahresgewinn und mehr als 10 Nutzer & 183,33\$ /Monat /Nutzer & 0\% & 5\% \\
		1'000'000\$+ Totaler Gewinn /Veröffentlichten Titel & Siehe oben gem. Anzahl Nutzer & 5\% auf den Gewinn nach 1'000'000\$ & 5\% \\
		\bottomrule 
	\end{tabular}
\end{table}

\par Bei den Optionen für das Programmieren werden verschiedene Programmiersprachen und Programmiertools angeboten. Je nach Programmiersprache ist so die mögliche Komplexität und Fehlerquote (Beispielsweise Memory-Management mit \emph{C++}) ein Hindernis für Einsteiger. Gleichzeitig bieten visuelle Lösungen Einstiegshilfe für weniger versierte Programmierer. Alle drei Optionen beinhalten solche Programmierumgebungen, wobei nur \emph{Unity} auch eine für \emph{Shaders}\footnote{Shader sind in diesem Kontext Softwaremodule, die Anzeigeeffekte für Computergrafiken implementieren.} anbietet.
\begin{table}[h]
	\centering
	\caption[Optionen für das Programmieren]{Optionen für das Programmieren}
	\label{tab:optionen_für_das_programmieren}
	\begin{tabular}{{p{0.3\linewidth} | p{0.15\linewidth} | p{0.25\linewidth} | p{0.15\linewidth}}}
		\toprule
		Type & Unity\cite{unity_introduction_2021}\cite{noauthor_unity_nodate} & Unreal Engine\cite{unreal_engine_blueprint_2021} & CryEngine\cite{cryengine_cryengine_2021-1} \\
		\midrule
		Programmiersprache & C\# & C++ & C++/Lua/C\# \\
		Andere Programmierarten & ShaderGraph, Visual Scripting  & Blueprints & FlowGraph \\
		\bottomrule 
	\end{tabular}
\end{table}

\newpage

\par Die Nutzung eines Produkts erfordert auch immer ein Wissen im Umgang damit. Dank Videoportalen, Blogs, Foren und vergleichbaren existieren heutzutage allerlei Auswahloptionen. Einerseits gibt es Offizielle Anleitungen/Schulungen, andererseits bieten oft Dritte dasselbe an.\footnote{Bei der Datenerhebung von Eigenen Anleitungen/Kurse wurden nicht einzelne Videos gezählt, sondern Ansammlungen davon. So zählt z.B. bei der CryEngine die Anleitung "Getting Started" als 1,  während "Animation" mit den 3 eigenen Subkategorien als 3 Einheiten zählt.}\footnote{Datenerfassen für "Dritte" wurde hier mittels Google Suche erstellt. Zur Verwertung wurden nur Sucheinträge unter der Kategorie 'Videos' erfasst. Die Suche selbst wurde mit "x tutorial" durchgeführt, wobei x $\in$ \{Unity, Unreal Engine, CryEngine\}. Zusätzlich wurden 10\% der Maximalen Anzahl Suchergebnisse Subtrahiert, um allfällige Falscheinträge zu vermindern. Gleichzeitig wurde auch darauf Verzichtet, jegliche Kostenpflichtige Kurse/Anleitungen in die Aufzählung mit ein zu beziehen.}

\begin{table}[h]
	\centering
	\caption[Wissensaneignung]{Wissensaneignung}
	\label{tab:wissensaneignung}
	\begin{tabular}{{p{0.3\linewidth} | p{0.15\linewidth} | p{0.25\linewidth} | p{0.15\linewidth}}}
		\toprule
		Herausgeber & Unity\cite{unity_learn_2021} & Unreal Engine\cite{unreal_engine_unreal_2021-1} & CryEngine\cite{cryengine_cryengine_2021-2} \\
		\midrule
		Eigene & 1'566 & 103 & 38 \\
		Drittanbietern & 17'370'000  & 663'300 & 30'960 \\
		\bottomrule 
	\end{tabular}
\end{table}

\newpage

Alle Optionen zur Wissensaneignung reichen aber nur bis zu einem gewissen Punkt. Entsprechend ist es relevant, wo man aktiv Hilfe erhält. Da hier von einer Kostenlosen Variante ausgegangen wird, ist eine Herstellerspezifische Direkthilfe ausgeschlossen.\footnote{Die Auswahl der Daten wurde anhand Programm-spezifischen Tasks auf StackOverflow und gamedev.stackexchange realisiert. Während der Inhalt der Fragen auch über die eingesetzten Programmiersprache sein könnte, sollten die Zahlen zeigen, wie aktiv eine Engine ist. Man kann zwar bei beiden Diensten die Suche weiter einschränken. Dies führt aber dazu, das bei gewissen Mengen ab 500 Resultate gefiltert wird und man dadurch keine sauberen Zahlen erhält.}

\begin{table}[h]
	\centering
	\caption[Handhabung von Problemen bei Unity]{Handhabung von Problemen bei Unity}
	\label{tab:handhabung_von_problemen_bei_unity}
	\begin{tabular}{c|c|c}
		\toprule
		Typ & gamedev.stackexchange.com\cite{game_development_stack_exchange_newest_nodate} & stackoverflow\cite{stack_overflow_newest_nodate}\\
		\midrule
		Gestellte Fragen & 13'927 & 60'977  \\
		\bottomrule 
	\end{tabular}
\end{table}

\begin{table}[h]
	\centering
	\caption[Handhabung von Problemen bei Unreal Engine]{Handhabung von Problemen bei Unreal Engine}
	\label{tab:handhabung_von_problemen_bei_unreal_engine}
	\begin{tabular}{c|c|c}
		\toprule
		Typ & gamedev.stackexchange.com\cite{game_development_stack_exchange_newest_nodate-1} & stackoverflow\cite{stack_overflow_newest_nodate-2}\\
		\midrule
		Gestellte Fragen & 621 & 1'819  \\
		\bottomrule 
	\end{tabular}
\end{table}


\begin{table}[h]
	\centering
	\caption[Handhabung von Problemen bei CryEngine]{Handhabung von Problemen bei CryEngine}
	\label{tab:handhabung_von_problemen_bei_cryengine}
	\begin{tabular}{c|c|c}
		\toprule
		Typ & gamedev.stackexchange.com\cite{game_development_stack_exchange_newest_nodate-2} & stackoverflow\cite{stack_overflow_newest_nodate-1}\\
		\midrule
		Gestellte Fragen & 30 & 35 \\
		\bottomrule 
	\end{tabular}
\end{table}


